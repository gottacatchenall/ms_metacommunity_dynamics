%!TEX TS-program = xelatex
\documentclass[10pt,oneside]{article}

\usepackage[english]{babel}

\usepackage{amsmath,amssymb,amsfonts}
\usepackage[utf8]{inputenc}
\usepackage[T1]{fontenc}
\usepackage{stix}
\usepackage[scaled]{helvet}
\usepackage[scaled]{inconsolata}

\usepackage{lastpage}

\usepackage{setspace}

\usepackage{ccicons}

\usepackage[hang,flushmargin]{footmisc}

\usepackage{geometry}

\setlength{\parindent}{0pt}
\setlength{\parskip}{6pt plus 2pt minus 1pt}

\usepackage{fancyhdr}
\renewcommand{\headrulewidth}{0pt}\providecommand{\tightlist}{%
  \setlength{\itemsep}{0pt}\setlength{\parskip}{0pt}}

\makeatletter
\newcounter{tableno}
\newenvironment{tablenos:no-prefix-table-caption}{
  \caption@ifcompatibility{}{
    \let\oldthetable\thetable
    \let\oldtheHtable\theHtable
    \renewcommand{\thetable}{tableno:\thetableno}
    \renewcommand{\theHtable}{tableno:\thetableno}
    \stepcounter{tableno}
    \captionsetup{labelformat=empty}
  }
}{
  \caption@ifcompatibility{}{
    \captionsetup{labelformat=default}
    \let\thetable\oldthetable
    \let\theHtable\oldtheHtable
    \addtocounter{table}{-1}
  }
}
\makeatother

\usepackage{array}
\newcommand{\PreserveBackslash}[1]{\let\temp=\\#1\let\\=\temp}
\let\PBS=\PreserveBackslash

\usepackage[breaklinks=true]{hyperref}
\hypersetup{colorlinks,%
citecolor=blue,%
filecolor=blue,%
linkcolor=blue,%
urlcolor=blue}
\usepackage{url}

\usepackage{caption}
\setcounter{secnumdepth}{0}
\usepackage{cleveref}

\usepackage{graphicx}
\makeatletter
\def\maxwidth{\ifdim\Gin@nat@width>\linewidth\linewidth
\else\Gin@nat@width\fi}
\makeatother
\let\Oldincludegraphics\includegraphics
\renewcommand{\includegraphics}[1]{\Oldincludegraphics[width=\maxwidth]{#1}}

\usepackage{longtable}
\usepackage{booktabs}

\usepackage{color}
\usepackage{fancyvrb}
\newcommand{\VerbBar}{|}
\newcommand{\VERB}{\Verb[commandchars=\\\{\}]}
\DefineVerbatimEnvironment{Highlighting}{Verbatim}{commandchars=\\\{\}}
% Add ',fontsize=\small' for more characters per line
\usepackage{framed}
\definecolor{shadecolor}{RGB}{248,248,248}
\newenvironment{Shaded}{\begin{snugshade}}{\end{snugshade}}
\newcommand{\KeywordTok}[1]{\textcolor[rgb]{0.13,0.29,0.53}{\textbf{#1}}}
\newcommand{\DataTypeTok}[1]{\textcolor[rgb]{0.13,0.29,0.53}{#1}}
\newcommand{\DecValTok}[1]{\textcolor[rgb]{0.00,0.00,0.81}{#1}}
\newcommand{\BaseNTok}[1]{\textcolor[rgb]{0.00,0.00,0.81}{#1}}
\newcommand{\FloatTok}[1]{\textcolor[rgb]{0.00,0.00,0.81}{#1}}
\newcommand{\ConstantTok}[1]{\textcolor[rgb]{0.00,0.00,0.00}{#1}}
\newcommand{\CharTok}[1]{\textcolor[rgb]{0.31,0.60,0.02}{#1}}
\newcommand{\SpecialCharTok}[1]{\textcolor[rgb]{0.00,0.00,0.00}{#1}}
\newcommand{\StringTok}[1]{\textcolor[rgb]{0.31,0.60,0.02}{#1}}
\newcommand{\VerbatimStringTok}[1]{\textcolor[rgb]{0.31,0.60,0.02}{#1}}
\newcommand{\SpecialStringTok}[1]{\textcolor[rgb]{0.31,0.60,0.02}{#1}}
\newcommand{\ImportTok}[1]{#1}
\newcommand{\CommentTok}[1]{\textcolor[rgb]{0.56,0.35,0.01}{\textit{#1}}}
\newcommand{\DocumentationTok}[1]{\textcolor[rgb]{0.56,0.35,0.01}{\textbf{\textit{#1}}}}
\newcommand{\AnnotationTok}[1]{\textcolor[rgb]{0.56,0.35,0.01}{\textbf{\textit{#1}}}}
\newcommand{\CommentVarTok}[1]{\textcolor[rgb]{0.56,0.35,0.01}{\textbf{\textit{#1}}}}
\newcommand{\OtherTok}[1]{\textcolor[rgb]{0.56,0.35,0.01}{#1}}
\newcommand{\FunctionTok}[1]{\textcolor[rgb]{0.00,0.00,0.00}{#1}}
\newcommand{\VariableTok}[1]{\textcolor[rgb]{0.00,0.00,0.00}{#1}}
\newcommand{\ControlFlowTok}[1]{\textcolor[rgb]{0.13,0.29,0.53}{\textbf{#1}}}
\newcommand{\OperatorTok}[1]{\textcolor[rgb]{0.81,0.36,0.00}{\textbf{#1}}}
\newcommand{\BuiltInTok}[1]{#1}
\newcommand{\ExtensionTok}[1]{#1}
\newcommand{\PreprocessorTok}[1]{\textcolor[rgb]{0.56,0.35,0.01}{\textit{#1}}}
\newcommand{\AttributeTok}[1]{\textcolor[rgb]{0.77,0.63,0.00}{#1}}
\newcommand{\RegionMarkerTok}[1]{#1}
\newcommand{\InformationTok}[1]{\textcolor[rgb]{0.56,0.35,0.01}{\textbf{\textit{#1}}}}
\newcommand{\WarningTok}[1]{\textcolor[rgb]{0.56,0.35,0.01}{\textbf{\textit{#1}}}}
\newcommand{\AlertTok}[1]{\textcolor[rgb]{0.94,0.16,0.16}{#1}}
\newcommand{\ErrorTok}[1]{\textcolor[rgb]{0.64,0.00,0.00}{\textbf{#1}}}
\newcommand{\NormalTok}[1]{#1}

\newlength{\cslhangindent}
\setlength{\cslhangindent}{1.5em}
\newlength{\csllabelwidth}
\setlength{\csllabelwidth}{3em}
\newenvironment{CSLReferences}[3] % #1 hanging-ident, #2 entry spacing
 {% don't indent paragraphs
  \setlength{\parindent}{0pt}
  % turn on hanging indent if param 1 is 1
  \ifodd #1 \everypar{\setlength{\hangindent}{\cslhangindent}}\ignorespaces\fi
  % set entry spacing
  \ifnum #2 > 0
  \setlength{\parskip}{#2\baselineskip}
  \fi
 }%
 {}
\usepackage{calc} % for \widthof, \maxof
\newcommand{\CSLBlock}[1]{#1\hfill\break}
\newcommand{\CSLLeftMargin}[1]{\parbox[t]{\maxof{\widthof{#1}}{\csllabelwidth}}{#1}}
\newcommand{\CSLRightInline}[1]{\parbox[t]{\linewidth}{#1}}
\newcommand{\CSLIndent}[1]{\hspace{\cslhangindent}#1}\usepackage[table,dvipsnames]{xcolor}

\geometry{includemp,
            letterpaper,
            top=1.2in,
            bottom=2.510cm,
            inner=0.5in,
            outer=0.4in,
            marginparwidth=1.95in,
            marginparsep=0.4in}

\usepackage[singlelinecheck=off]{caption}
\captionsetup{
  font={small},
  labelfont={bf},
  format=plain,
  labelsep=quad
}
\usepackage{floatrow}
\floatsetup[figure]{margins=hangright,
              facing=no,
              capposition=beside,
              capbesideposition={center,outside},
              floatwidth=\textwidth}
\floatsetup[table]{margins=hangoutside,
             facing=yes,
             capposition=beside,
             capbesideposition={center,outside},
             floatwidth=\textwidth}

\pagestyle{plain}

\setcounter{secnumdepth}{5}

\usepackage{titlesec}

\titleformat{\section}[block]
{\normalfont\large\sffamily}
{\thesection}{.5em}{\titlerule\\[.8ex]\bfseries}

\titleformat{\subsection}[runin]
{\normalfont\fontseries{b}\selectfont\filright\sffamily}
{\thesubsection.}{.5em}{}

\titleformat{\subsubsection}[runin]
{\normalfont\itshape\sffamily}{(\thesubsubsection)}{1em}{}

\fancypagestyle{firstpage}
{
   \fancyhf{}
   \renewcommand{\headrulewidth}{0pt}
   \fancyfoot[R]{\footnotesize\ccby}
   \fancyfoot[L]{\footnotesize\sffamily\today}
}

\fancypagestyle{normal}
{
  \fancyhf{}
  \fancyfoot[R]{\footnotesize\sffamily\thepage\ of \pageref*{LastPage}}
}

\usepackage{tikz}
\begin{document}
\tikz [remember picture, overlay] %
\node [shift={(-0.6in,1.1cm)},scale=0.2,opacity=0.4] at (current page.south east)[anchor=south east]{ };%
\pagestyle{normal}
\thispagestyle{firstpage}

\newcommand{\colorRule}[3][black]{\textcolor[HTML]{#1}{\rule{#2}{#3}}}

\noindent {\LARGE \textbf{\textsf{MetacommunityDyanmics.jl: A virtual
laboratory for simulating species interaction networks across space and
time}}}

\medskip
\begin{flushleft}
{\small
%
\href{https://orcid.org/0000-0002-6506-6487}{M.D.\,Catchen}%
%
\,\textsuperscript{1,2}
\vskip 1em
\textsuperscript{1}\,McGill University; \textsuperscript{2}\,Québec
Centre for Biodiversity Sciences\\
\vskip 1em
\textbf{Correspondance to:}\\
}
\end{flushleft}

\vskip 2em
\makebox[0pt][l]{\colorRule[CCCCCC]{2.0\textwidth}{0.5pt}}
\vskip 2em
\noindent

\marginpar{\vskip 1em\flushright
{\small{\bfseries Keywords}:\par
pandoc\\pandoc-crossref\\github actions\\}
}


\textbf{Abstract}:\,A work in progress manuscript.

\vskip 2em
\makebox[0pt][l]{\colorRule[CCCCCC]{2.0\textwidth}{0.5pt}}
\vskip 2em

MetacommunityDynamics.jl Manuscript -- outline

\hypertarget{introduction}{%
\section{Introduction}\label{introduction}}

\begin{itemize}
\tightlist
\item
  What is a metacommunity?
\item
  Why do we need software to simulate metacommunities?
\item
  What can other people do with this software?
\item
  What does the rest of this paper look like?
\end{itemize}

\hypertarget{methods}{%
\section{Methods}\label{methods}}

\hypertarget{outline-of-theoretical-framework-used-to-represent-metacommunity-dynamics}{%
\subsection{Outline of theoretical framework used to represent
metacommunity
dynamics}\label{outline-of-theoretical-framework-used-to-represent-metacommunity-dynamics}}

\begin{itemize}
\tightlist
\item
  Metacommunity paradigms (Species Sorting, Mass-Effect, Neutral, Patch
  Dynamics)
\item
  Velland 2010 --- fundemental mechanisms (dispersal, speciation,
  selection, drift)
\end{itemize}

\hypertarget{outline-of-software-structure}{%
\subsection{Outline of software
structure}\label{outline-of-software-structure}}

\hypertarget{the-species-pool-and-the-metaweb}{%
\subsubsection{The species pool and the
metaweb}\label{the-species-pool-and-the-metaweb}}

\hypertarget{landscape-and-dispersal-structure}{%
\subsubsection{Landscape and dispersal
structure}\label{landscape-and-dispersal-structure}}

\hypertarget{the-metacommunity-tensor}{%
\subsubsection{The Metacommunity
Tensor}\label{the-metacommunity-tensor}}

\begin{itemize}
\tightlist
\item
  Abstract dynamics model / difference map
\item
  space/time/species axes
\end{itemize}

\hypertarget{results}{%
\section{Results}\label{results}}

Use-case examples.

\begin{enumerate}
\def\labelenumi{\arabic{enumi}.}
\tightlist
\item
  Co-evolution of plant-pollinator interaction netowrks
\item
  Occupancy of food-webs in a landscape
\item
  Invasive vs.~native plant in a landscape
\item
  How does landscape structure effect species richness?
\item
  Disease spillover interaction networks
\end{enumerate}

\hypertarget{discussion}{%
\section{Discussion}\label{discussion}}

What next for the software?

\begin{itemize}
\tightlist
\item
  parameter estimation from data
\item
  scalable/parallelizable
\end{itemize}

How can this be applied by others

\hypertarget{references}{%
\section{References}\label{references}}

\end{document}
